
\chapter{Literature Survey and Motivation}
In the realm of cloud storage security, a comprehensive literature survey reveals the evolving
landscape of encryption methodologies and the persistent challenges in safeguarding user data.
Existing studies have explored conventional encryption techniques, emphasizing the need for one-to-many encryption, data integrity assurance, and resilient data deletion to counteract potential
vulnerabilities. Identity-based encryption (IBE), attribute-based encryption (ABE), homomorphic
encryption, and searchable encryption have been examined for their efficacy in fortifying cloud
environments. Notably, the integration of a load balancer with GitHub as a strategic approach to
data management is a novel contribution to the literature, optimizing resource utilization and
ensuring balanced data distribution.
Furthermore, the literature highlights the increasing relevance of post-quantum encryption in
response to emerging threats. The principles of IBE, ABE, homomorphic encryption, and searchable
encryption have been studied extensively, offering valuable insights into potential new encryption
models. However, gaps in the literature underscore the need for innovative solutions that precisely
align with identified security requirements, particularly in the context of the dynamic cloud storage
environment.
The motivation behind this research stems from the critical importance of addressing and
overcoming the identified gaps in current cloud storage security measures. As cloud storage
continues to be a pivotal component of modern data management, ensuring robust data security and
privacy protection is imperative. The exploration of diverse encryption methodologies and the
strategic integration of a load balancer with GitHub present an opportunity to contribute novel
solutions to the existing body of knowledge. The motivation to delve into post-quantum encryption
is fueled by the recognition of evolving threats, emphasizing the need for adaptive and advanced
security measures.
The innovative use of a load balancer in conjunction with GitHub serves as a motivation derived
from practical considerations. The approach of creating multiple repositories for a single user is
driven by the desire to optimize resource utilization and achieve a balanced distribution of data,
addressing potential vulnerabilities in current data management practices. This motivation is
grounded in the belief that efficient data distribution not only enhances security but also contributes
to the overall efficiency and reliability of cloud storage systems.
Moreover, the motivation to explore post-quantum encryption arises from a forward-looking
perspective. The anticipation of emerging threats in the post-quantum era propels the research
towards investigating encryption models that can withstand evolving challenges. By delving into
the principles of IBE, ABE, homomorphic encryption, and searchable encryption, the research aims
to provide a foundation for potential new encryption models that align precisely with the dynamic
security requirements of cloud storage.
In conclusion, the motivation for this research is multifaceted – it is driven by the desire to address
current gaps in literature, contribute innovative solutions to practical challenges in data
management, and proactively tackle emerging threats through the exploration of advanced
encryption models. This research aspires to make a significant impact on the field by ensuring the
continuous evolution of cloud storage security in response to the dynamic nature of modern security
challenges. 
\clearpage